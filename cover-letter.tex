%
\documentclass[a4paper,11pt]{article}
\usepackage[T1]{fontenc}
%\usepackage{polski}
\usepackage[utf8x]{inputenc}
\usepackage{microtype}

\usepackage{times}

\usepackage[
pdftitle={Letter},
colorlinks=true,linkcolor=black,urlcolor=black,citecolor=black]{hyperref}
\urlstyle{same}

\usepackage{geometry}
\geometry{left=25mm,right=25mm,%
bindingoffset=0mm, top=25mm,bottom=25mm}

\linespread{1.3}
\pagestyle{empty}


\setlength{\parindent}{0em}
\setlength{\parskip}{1em}
\renewcommand{\baselinestretch}{1.5}

\begin{document}

\vskip 2.0cm

\phantom{a}\hfill
\begin{tabular}[c]{@{}l@{}}
{\bf Luiz Alberto Ferreira Gomes}\\
University of Campinas\\ 
Pontifical Catholic University of Minas Gerais\\
luizgomes@pucpcaldas.br\\
\noalign{\vskip 2mm}  

{\bf Ricardo da Silva Torres}\\
Norwegian University of Science and Technology\\
ricardo.torres@ntnu.no\\
\noalign{\vskip 2mm} 

{\bf Mario Lúcio Côrtes}\\
University of Campinas\\
cortes@ic.unicamp.br

\end{tabular}

\vskip 2.0cm

{\setlength{\parindent}{0cm}
\today
}

\vskip 0.5cm

\noindent
Dear Paris Avgeriou, 

\noindent
We wish to submit an original research article entitled ``BERT-based Feature Extraction for Long-lived Bug Prediction in FLOSS'' for consideration by 
Journal of Systems and Software. 

\noindent
We confirm that this work is original and has not been published elsewhere, nor is it currently under consideration for publication elsewhere.

\noindent
In this paper, we show that long-lived bug prediction using BERT-based feature extraction systematically outperformed the traditional TF-IDF in Free/Libre Open Source Software (FLOSS) projects. This is significant because it demonstrates a promising avenue to predict long-lived bugs based on BERT contextual embeddings features. We believe that this manuscript is appropriate for publication by Journal of Systems and Software because it related to the topic {\bf artificial intelligence, data analytics and big data applied in software engineering}. 

%\noindent
%{\it [Please explain in your own words the significance and novelty of the work, the problem that is being addressed, and why the manuscript belongs in this journal. Do not simply insert your abstract into your cover letter! Briefly describe the research you are reporting in your paper, why it is important, and why you think the readership of the journal would be interested in it.]}

Estimating fixing times and the immediate identification of opened bugs that may have a long-lived lifecycle is essential for maintenance and quality teams to build their working plan. Managers in charge of maintenance may improve the resource allocation and better create a release roadmap avoiding the bugs backlog overgrow, especially in FLOSS context. Besides, the correct prediction of long-lived bugs may help the quality assurance team improve their daily activities. They could fix more bugs that often adversely affect software quality and disturb the user experience across versions of FLOSS, which may increase customers’ satisfaction, avoiding them to switch in favor of a competitor. 

\noindent
In our paper we collected bug reports from six popular FLOSS projects repositories (Eclipse, Freedesktop, Gnome, GCC, Mozilla, and WineHQ) and used the following machine learning classifiers to predict long-lived bugs: K-Nearest Neighbor, Naïve Bayes, Neural Networks, Random Forest, and Support Vector Machines. Furthermore, we compare different feature extractors, based on BERT and TF-IDF methods, in the construction of  long-lived prediction models. 

\noindent
We strongly believe that the journal's readers would be interested in our paper because our research points out a promising avenue to predict long-lived bugs, which combines well-known machine learning methods with state-of-the-art NLP methods.

\noindent
We have no conflicts of interest to disclose. 

\noindent
Please forward all correspondence concerning this manuscript to me at gomes.luiz@gmail.com.

\noindent
Thank you for your consideration of this manuscript. 
\noindent
Sincerely,

\noindent
Luiz Alberto Ferreira Gomes
\vskip 2.0cm


\end{document}
 