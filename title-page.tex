%
\documentclass[a4paper,11pt]{article}
\usepackage[T1]{fontenc}
%\usepackage{polski}
\usepackage[utf8x]{inputenc}
\usepackage{microtype}

\usepackage{times}

\usepackage[
pdftitle={Letter},
colorlinks=true,linkcolor=black,urlcolor=black,citecolor=black]{hyperref}
\urlstyle{same}

\usepackage{geometry}
\geometry{left=25mm,right=25mm,%
bindingoffset=0mm, top=25mm,bottom=25mm}

\linespread{1.3}
\pagestyle{empty}


\setlength{\parindent}{0em}
\setlength{\parskip}{1em}
\renewcommand{\baselinestretch}{1.5}

\begin{document}

\vskip 2.0cm
  
\begin{center}
    {\bf BERT-based Feature Extraction for Long-lived Bug Prediction in FLOSS} 
    
    \vskip 2.0cm
    {\bf Luiz Alberto Ferreira Gomes}\\
    Ph.D in Computer Science\\
    Assistant Professor\\
    Institute of Computing, University of Campinas, Campinas, São Paulo\\ Institute of Exact Sciences and Informatics, Pontifical Catholic University of Minas Gerais, Belo Horizonte, Minas Gerais\\
    luizgomes@pucpcaldas.br
    
    \vskip 0.5cm
    {\bf Ricardo da Silva Torres}\\
    Ph.D in Computer Science\\
    Professor\\
    Department of ICT and Natural Sciences, NTNU -- Norwegian University of Science and Technology, {\AA}lesund, Norway\\
    ricardo.torres@ntnu.no
    
    \vskip 0.5cm
    {\bf Mario Lúcio Côrtes}\\
    Ph.D in Computer Science\\
    Associate Professor\\
    Institute of Computing, University of Campinas, Campinas, São Paulo\\
    cortes@ic.unicamp.br
    

\end{center}


\end{document}
 